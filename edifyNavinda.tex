\documentclass{article}
\usepackage{graphicx}
\usepackage{amsmath,amssymb}
\usepackage[numbers]{natbib}
\usepackage{booktabs}
\usepackage{csquotes}
\usepackage{afterpage}

\title{SmartGuard: An IoT Model to Reduce Human Monkey Conflict in Sri Lanka}
\author{Navinda Hewawickrama}
\date{October 2025}

\begin{document}

\maketitle
The boundary between humans and animals in Sri Lanka is always conflicting as the landscape evolves rapidly. Along with elephants in some areas, the toque macaque is always fighting with humans. This is created by humans populating and farming in their natural habitats and the expansion of peri-urban areas. A lot of damages for households and smallholder farmers, such as destroyed gardens and crops, property damage, and the very real risk of disease transmission or bites, are always there. Standard solutions have not worked. Constant human protection cannot be done all the time, and as intelligent primates get used to these ineffective threats, simple tactics like scarecrows or firecrackers quickly lose their effectiveness. Electric fencing and other physical barriers are too expensive and impractical and complicated for village homes.  There is a need for a method that is as intelligent and flexible as the monkeys themselves because the costs of lethal control are expensive. This calls for an evolutionary shift away from conflict and towards managed coexistence, using modern technology to restore a sustainable border.
\\ \\
The idea we have is SmartGuard, a system designed as a community-based solution rather than a standalone product.  The five guiding principles of its core philosophy are intended to be flexible and effective.  In order to create a proactive rather than reactive approach, the first is early detection, which shifts the point of intervention from the raid to the approach.  In order to break the cycle of routine, the second strategy is adaptive deterrence, which makes sure that the system's reactions are unpredictable and diverse. The third concept is offline operation, a strategy that ensures functionality even in places with weak or poor internet access.  The fourth is community coordination, which turns the response from unorganised, wild acts into a coordinated defence at the neighbourhood level.  Lastly, the system is made of locally serviceable parts, is solar-powered, is tolerant and effective in tropical conditions, and is readily available and robust.  Simply SmartGuard is a distributed, low-power network of intelligence that acts as a guardian in the village, protecting human lives while making sure humans are not disturbing primates' natural lifestyle.
\\ \\
Every part of Smart Guard is created for low power consumption, high performance, and local decision-making so that it can work as a single integrated digital ecosystem. Every node has a sensor fusion core, a multi-layered detection method that allows it to avoid false alarms and uses less energy. The first thing that happens is an infrared or thermal signal will act as the first tripwire, which serves as a wake-up signal when it detects motion or a heat signature. A sensor that records brief audio bursts is activated by this. The unique calls of macaques are separated from other background noises, such as dogs, birds, or human activity, by an on-device model. A camera with low-resolution monochrome (infrared camera) is used for final confirmation.  In addition to protecting resident privacy and reducing the computational and power burden, this decision is essential because it offers enough visual data for precise species classification. Without processing the data faraway, local processing can be done on this fused sensor.
\\ \\
The SmartGuard device has a lot of nodes which act as its own on-device artificial intelligence brain. A power-efficient microcontroller unit such as an ESP32-S3 runs a quantized, compact convolutional neural network. This small AI model can use the sensor data and classify whether it is a primate or not on the board itself. This quick processing makes the system's real-time reaction very fast, eliminating the latency and dependency of a cloud approach. After identifying correctly, the node activates its "nervous system", a low-power, long-range communication protocol. The LoRaWAN acts as the primary backbone, which gives the option to the node to transmit an extremely small data packet of less than two hundred bytes over a distance of several kilometres to a central community gateway on infinitesimal amounts of power. As a standby in scenarios where coverage is strong, NB-IoT or LTE-M cellular technology can be used. When internet connections are low, Bluetooth can also be implemented as a backup where local provisioning and maintenance using a mobile app such that the node can be managed by local technicians directly.
\\ \\
The real innovation here, in the SmartGuard, is not just the detection capabilities of this small device but also in its sophisticated approach to deterrence, which is specifically engineered to overcome the primates' capacity for creating habitats in the environment. Creating a response that is static and does not change will definitely fail. So, each SmartGuard node is given a toolkit of non-lethal actuators, and the system's central logic is programmed to deploy them in a dynamic and unpredictable sequence. The deterrent package has a set of directional audio projectors that can project a range of distracting sounds at the approaching troop without bothering the entire village, such as recorded animal sounds, ultrasonic echoes, or the clanging of metal. A smart light strobe system which will provide high-intensity, eye-safe flashes, with the system automatically changing their intensity based on the time of day to use the startling effect more effectively. Non-harmful water jets will be sprayed, which is a universally aversive stimulus for these animals, with a built-in rain sensor to lock out the mechanism during wet weather and conserve water.
\\ \\
The main logic is the anti-habituation logic in this toolkit. The system is designed to never respond to the monkeys in the same way twice. It changes its type of methods intelligently and randomly, the sequence of their activation, and their intensity. More importantly, it operates based on the scale of the monkey group. The central gateway manages a coordinated response as a monkey troop passes through the village, activating various repellents from various nodes along their journey path.  This gives the animals the impression that the entire defended area,rather than just a single property, is dangerous. This community-wide, adaptive defence is key to teaching the troops to avoid the human settlement altogether, rather than simply learning to bypass one household’s scarecrow.
\\ \\
But without a smooth transition into the human society that it is used to help, even the most advanced technology is useless.  The main and important framework that turns a collection of devices into a long-term solution is the community operations model. The village uses a cheap computer, like a Raspberry Pi, running user-friendly programming software, like Node-RED, as its operations room. Users who are authorized  can access a real-time heat map of animal movements created by this central hub, which collects and processes all incoming data packets from the sensor nodes. Based on a troop's current journey path and historical movement patterns, the hub can use basic predictive algorithms to predict the troop's likely path for the next 30 to 60 minutes.  The system can effectively defend the area the troop is likely to enter next by issuing automated \enquote{playbooks}, which are predefined methods and sequences to chase the monkeys away, thanks to this predictive capability.
\\ \\
The user interface for this is a simple mobile application that can work in multiple languages, which should be available in Sinhala, English and Tamil. This is used to help the local people, not to scare them. Instead of using huge alarm sounds, it delivers calm, actionable notifications, such as, \enquote{Troop approaching from the southern tree line; secure loose food items}. This changes the panicking environment to a passive one where the people can also actively engage in the coordinated defence strategy. Crucially, this deployment should be a co-design process, developed in close collaboration with local leaders, farmer societies, and residents to ensure deep community buy-in and create the system to help and work well with the local needs and knowledge.
 
The pilot phase is mainly a learning period, where data on primate behaviour and system performance are used to continuously correct predictive models and expand libraries of how to chase away the animals.  SmartGuard's sustainability, however, is dependent on how well it is integrated into the local economy and knowledge base.  This means training a team of local technicians that can perform routine maintenance and troubleshooting, and also making sure a spare-parts backup is always readily available with components to ensure repairs can be completed quickly and affordably. The system's robust design is a direct investment in long-term community ownership.
\\ \\
SmartGuard's success will be measured using a framework of clear, tangible metrics centred on how well it will impact the real world.  The primary indicator will be a 50 per cent reduction in confirmed monkey attacks within the defended zones during the first operational year.  In addition, quantifying the economic benefit by tracking the money saved from saving the food, crop and property damage presents a clear and compelling case for the investment.  Most importantly, an important rule and key performance indicator is the guarantee of zero instances of wildlife harm, which makes sure that the solution stays true to its non-lethal harmonious existence concept. SmartGuard's guiding ethic is not eliminating the animals or dominating them, but rather creating a new, intelligent environment.  It is a tool for changing how humans engage with animals, using non-invasive technology to protect humans while completely respecting these intelligent animals in Sri Lanka's ecosystem. It is a promising model for the future of the humans and animals for the harmony of both species.
\end{document}